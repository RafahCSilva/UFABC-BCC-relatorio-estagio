\documentclass[
  % -- opções da classe memoir --
  12pt,				% tamanho da fonte
  %openright,			% capítulos começam em pág ímpar (insere página vazia caso preciso)
  openany,
  %twoside,			% para impressão em verso e anverso. Oposto a oneside
  oneside,
  %titlepage,
  a4paper,			% tamanho do papel.
  % -- opções da classe abntex2 --
  %chapter=TITLE,		% títulos de capítulos convertidos em letras maiúsculas
  %section=TITLE,		% títulos de seções convertidos em letras maiúsculas
  %subsection=TITLE,	% títulos de subseções convertidos em letras maiúsculas
  %subsubsection=TITLE,% títulos de subsubseções convertidos em letras maiúsculas
  % -- opções do pacote babel --
  english,			% idioma adicional para hifenização
  brazil
]{article}
\usepackage[utf8]{inputenc}
\usepackage[brazil]{babel}
\usepackage{amsfonts}
\usepackage{indentfirst}
\usepackage[T1]{fontenc}
\usepackage{helvet}
\renewcommand{\familydefault}{\sfdefault}
\usepackage[left=3cm, right=2cm, top=3cm, bottom=2cm]{geometry}
\usepackage{setspace}
\usepackage{epigraph}
\usepackage{graphicx}
\usepackage{hyperref}
\usepackage{amsmath}
\numberwithin{figure}{section}
\numberwithin{table}{section}

\graphicspath{{figuras/}}

\renewcommand{\baselinestretch}{1.5}
\setlength{\parindent}{3em}
\setlength{\parskip}{1em}

%% DEFINEs dos textos constantes da capa
\newcommand{\defUFABC}{UNIVERSIDADE FEDERAL DO ABC}
\newcommand{\defBCC}{BACHARELADO EM CIÊNCIA DA COMPUTAÇÃO}
\newcommand{\defRelatorio}{RELATÓRIO DO ESTÁGIO SUPERVISIONADO I} %%ESTÁGIO SUPERVISIONADO
\newcommand{\defTitulo}{ESTÁGIO EM DESENVOLVIMENTO DE SISTEMA PARA USO INTERNO NA AGÊNCIA CADARIS}
\newcommand{\defRafael}{RAFAEL CARDOSO DA SILVA}
\newcommand{\defOrientador}{Orientador: Prof. Dr. Daniel Morgato Martin}
\newcommand{\defLocaldata}{Santo André, 2017}

%%%%% \subsubsubsection
\usepackage{titlesec}
\usepackage{hyperref}
\titleclass{\subsubsubsection}{straight}[\subsection]

\newcounter{subsubsubsection}[subsubsection]
\renewcommand\thesubsubsubsection{\thesubsubsection.\arabic{subsubsubsection}}
\renewcommand\theparagraph{\thesubsubsubsection.\arabic{paragraph}} % optional; useful if paragraphs are to be numbered

\titleformat{\subsubsubsection}
{\normalfont\normalsize\bfseries}{\thesubsubsubsection}{1em}{}
\titlespacing*{\subsubsubsection}
{0pt}{3.25ex plus 1ex minus .2ex}{1.5ex plus .2ex}

\makeatletter
\renewcommand\paragraph{\@startsection{paragraph}{5}{\z@}%
  {3.25ex \@plus1ex \@minus.2ex}%
  {-1em}%
  {\normalfont\normalsize\bfseries}}
\renewcommand\subparagraph{\@startsection{subparagraph}{6}{\parindent}%
  {3.25ex \@plus1ex \@minus .2ex}%
  {-1em}%
  {\normalfont\normalsize\bfseries}}
\def\toclevel@subsubsubsection{4}
\def\toclevel@paragraph{5}
\def\toclevel@paragraph{6}
\def\l@subsubsubsection{\@dottedtocline{4}{7em}{4em}}
\def\l@paragraph{\@dottedtocline{5}{10em}{5em}}
\def\l@subparagraph{\@dottedtocline{6}{14em}{6em}}
\makeatother

\setcounter{secnumdepth}{4}
\setcounter{tocdepth}{4}



\begin{document}

%%--[ELEMENTOS PRÉ-TEXTUAIS]--%%

%%--CAPA--%%

\begin{titlepage}

\begin{center}
\text{ \defUFABC }
\text{ \defBCC }\\\vspace{5cm}
\textbf{\large{ \defRelatorio }}\\\vspace{4cm}
\textbf{\Large{\defTitulo}}\\\vspace{4cm}
% Nome do Autor
\textbf{\large{ \defRafael }}\\\vspace{2cm}
% Orientador
\textbf{\large{ \defOrientador }}\\\vspace{3cm}
\textbf{\defLocaldata}
\end{center}

\end{titlepage}

%%--FOLHA DE ROSTO--%%

\begin{titlepage}

\begin{center}
\textbf{\large{ \defRafael }}\\\vspace{6cm}
\textbf{\Large{ \defTitulo }}\\\vspace{5cm}
\begin{raggedleft}{
Relatório do Estágio apresentado ao Curso de\\
Bacharelado em Ciência da Computação como\\
requisito parcial para obtenção do grau de\\
Bacharel em Ciência da Computação
\\\vspace{1cm}

%  Trabalho apresentado à banca de Estágio\\
%  Supervisionado como requisito parcial para\\
%  obtenção do título de Bacharel em Ciência da\\
%  Computação pela Universidade Federal do ABC.\\
\large{ \defOrientador }
}\\\vspace{5cm}
\end{raggedleft}
\textbf{ \defLocaldata }
\end{center}

\end{titlepage}

%%--DEDICATÓRIA--%%

\begin{titlepage}

\begin{center}
\textbf{DEDICATÓRIA}
\end{center}

Dedico todo projeto realizado aos meus amigos e familiares que depositaram grande confiança em mim desde o início.


\end{titlepage}

%%--AGRADECIMENTOS--%%

\begin{titlepage}

\begin{center}
\textbf{AGRADECIMENTOS}
\end{center}

Meus agradecimentos são voltados à agência Cadaris, principalmente a Diretora Comercial, que propôs essa oportunidade de amadurecer meus conhecimentos obtidos na Universidade Federal do ABC, além do crescimento pessoal.


\end{titlepage}

%%--EPÍGRAFE--%%

%\begin{titlepage}
%.\\\vspace{18cm}
%\begin{raggedleft}
%
%\begin{epigraph} {????????????????????? \\A maioria das pessoas pensa no sucesso e no fracasso como opostos, mas eles são ambos produtos do mesmo processo}{Roger Von Oech}
%
%\end{epigraph}
%\end{raggedleft}
%
%\end{titlepage}

%%--RESUMO--%%

%\begin{titlepage}
%
%\begin{center}
%\textbf{RESUMO}
%\end{center}
%
%% TODO: RESUMO
%
%????????????????? DEPOIS ?????????????????
%
%%O estágio consiste na manutenção de sistemas existentes em PHP, tanto nas permissões de acesso de usuários quanto segurança das informações. Geralmente são correções e adição de informações nas funcionalidades já existentes. Novos sistemas são desenvolvidos com o framework .NET, banco de dados SQL Server e padrão MVC. A fim de melhorar a experiência do desenvolvimento e do usuário final, ferramentas modernas são exploradas, como o AngularJS by Google. O sistema de bolsa de estudos foi o primeiro com a ferramenta e, após êxito no desenvolvimento e nos testes, foi iniciado o sistema de gerenciamento de projetos com exploração máxima dos recursos, com mais agilidade e padrão no desenvolvimento e uma experiência mais ativa e dinâmica por parte do usuário final.
%
%
%\end{titlepage}

%%--ABSTRACT--%%

%\begin{titlepage}
%
%\begin{center}
%\textbf{ABSTRACT}
%\end{center}
%% TODO: RESUMO ingles
%
%????????????????? DEPOIS ?????????????????
%
%%The internship is the maintenance of existing systems in PHP, both on user access permissions as information security. Usually fixes and adding information to existing functionality. New systems are developed with the .NET framework, SQL Server database and MVC pattern. To enhance the experience of development and user, modern tools are explored, such as angularjs by Google. The scholarship system was the first with the tool and, after successful development and testing, started the project management system with maximum exploitation of resources, more quickly and standard development and a more active and dynamic experience by the user.
%
%
%\end{titlepage}


%%%--LISTA DE FIGURAS--%%
%
%\begin{titlepage}
%
%\begin{center}
%  \begin{singlespace}
%    \listoffigures
%  \end{singlespace}
%\end{center}
%
%\end{titlepage}


%%--SUMÁRIO--%%

\begin{titlepage}

\begin{center}
  \begin{singlespace}
    \tableofcontents
  \end{singlespace}
\end{center}

\end{titlepage}


%%--[ELEMENTOS TEXTUAIS]--%%

\section{INTRODUÇÃO}

O estágio é uma atividade fundamental para a formação do aluno que está matriculado em um curso de nível superior, já que o discente tem a possibilidade de fazer uma aliança entre os conhecimentos adquiridos durante o período de graduação com a experiência vivencial no ambiente corporativo. 

Além de agregar a responsabilidade de ter uma profissão, o estágio permite que o aluno desenvolva habilidades que são essenciais no mercado de trabalho, tais como liderança, trabalhar em equipe, resiliência, entre outros, através de situações e desafios que possam vir ocorrer dentro da organização e de suas responsabilidades.

O estágio é uma tarefa supervisionada por um orientador, e a descrição e avaliação do aluno quanto às suas atividades, conhecimentos adquiridos e habilidades desenvolvidas são relatados em um relatório.

Este relatório tem por objetivo descrever as atividades e os produtos obtidos durante a disciplina de Estágio Supervisionado I. Nesta seção, apresenta-se a caracterização do estágio, da empresa, uma visão geral do estágio e a organização geral deste documento.



\subsection{CARACTERÍSTICAS DO ESTÁGIO}

A modalidade deste estágio é em \textit{Home Office}\footnote{Escritório em casa, em uma tradução livre do inglês, trabalho que é realizado em espaço alternativo ao escritório de uma empresa.}, que também pode ser definido com um trabalho remoto ou teletrabalho, para a Cadaris Comunicação \textit{[ref:Cadaris]}. A jornada de trabalho é de segunda à sexta-feira, desempenhando 6 horas diárias, totalizando 30 horas semanais, com a gratificação de uma bolsa de R\$ 1.100,00 por mês.

Por se tratar de um \textit{Home Office}, proporcionou ao discente um horário flexível de trabalho fazendo com que ele tenha autonomia para conciliar seu trabalho com as suas outras atividades. A comunicação com a empresa foi mantida através de e-mails e ligações via telefone e \textit{Skype}.

O objetivo geral deste estágio é o desenvolvimento de um projeto de uso interno para a empresa.




\subsection{CARACTERIZAÇÃO E ANÁLISE DA EMPRESA}

%  A Cadaris é uma agência de publicidade com atuação nas áreas de marketing, editorial e comunicação. Aqui, trabalho em equipe é lei e surpreender o cliente, uma meta diária.
%  Trabalhamos com ética e transparência, de forma responsável e comprometida. Produzimos com qualidade e criatividade, sempre com respeito aos prazos acordados.
%  Queremos ser reconhecidos como uma agência diferenciada, por clientes e funcionários.
%  Entre os nossos principais clientes estão: Colgate-Palmolive, Biotropic, ABO, TAM e Mattel, entre outras empresas.
%  Para saber mais sobre a Cadaris, acesse www.cadaris.com.br ou envie um email para cadaris@cadaris.com.br

%  Products:
%    Publicidade: anúncios, materiais de ponto de venda, materiais promocionais, apresentações de produtos e campanhas, hotsites, etc.
%    Comunicação: campanhas dirigidas e campanhas de comunicação interna
%    Editorial: revistas, informativos, enews, jornal-mural, guias e relatórios, manuais, cartilhas, etc.
%  ONDE A GENTE ATUA
%  EDITORIAL
%    Desenvolvimento de conteúdos editoriais baseado em planejamento, organização e processos de jornalismo.
%  ESTRATÉGIA
%    Criação de campanhas e peças de comunicação dirigida, comunicação interna, incentivo de vendas e marketing de relacionamento.
%  PUBLICIDADE
%    Assertividade em briefing, agilidade e criatividade são os drivers na criação das peças publicitárias e promocionais.
%  WEB
%    Levantamento de requisitos e necessidades, arquitetura da informação, design de projeto e desenvolvimento web.


O estágio foi iniciado no dia 01 de Agosto de 2016, contratado pela Cadaris Comunicações, CNPJ 01.556.009/0001-46, que é uma agência de publicidade e está localizada na R. Dr. Thirso Martins, 100, cj 303, CEP:~04120-050, Vila Mariana, São Paulo~-~SP, telefone:~(11)~5571-9142.

Fundada em 11 de novembro de 1996, pelos sócios Maristela Harada e Frederico Pimenta, a Cadaris Comunicação é formada por profissionais de diferentes áreas da comunicação, o que proporciona mais eficácia e agilidade. Trabalhando com ética e transparência, de forma responsável e comprometida. Produzindo com qualidade e criatividade, sempre com respeito aos prazos acordados.

A Cadaris é uma agência de comunicação com atuação nas áreas de:
\vspace{-0.5cm}

\begin{description}
   \item [EDITORIAL] Desenvolvimento de conteúdos editoriais baseado em planejamento, organização e processos de jornalismo, como produtos: revistas, informativos, enews, jornal-mural, guias e relatórios, manuais, cartilhas, etc.
   \item [ESTRATÉGIA] Criação de campanhas e peças de comunicação dirigida, comunicação interna, incentivo de vendas e marketing de relacionamento.
   \item [PUBLICIDADE]  Assertividade em briefing, agilidade e criatividade, produzindo: Anúncios, materiais de ponto de venda, materiais promocionais, apresentações de produtos e campanhas, hotsites, etc.
   \item [WEB] Levantamento de requisitos e necessidades, arquitetura da informação, design de projeto e desenvolvimento web.
\end{description}

Entre os principais clientes da Cadaris estão: Colgate-Palmolive, Biotropic, ABO, TAM e Mattel.


\subsection{VISÃO GERAL DO ESTÁGIO}


Durante o primeiro mês, o estagiário foi introduzido ao funcionamento da agência com a familiarização das regras do negócio e foram apresentadas as ferramentas atuais utilizadas para apoiar a administração da empresa. A fim de desenvolver um \textit{software} de Sistema de Gestão Empresarial substituto, para uso interno da Cadaris.

Após este período, foi feito o estudo das tecnologias que seriam utilizadas para o desenvolvimento deste sistema proposto que já estavam sendo discutidas desde o início do estágio.

Em seguida, iniciou-se a implementação do sistema com foco inicial no \textit{Back-end} que foram seguidas em três etapas: Desenvolvimento do Cadastro de Recursos e algumas rotinas administrativas; Processo Comercial; Geração de Relatórios e Controle Financeiro.

Por fim, foi desenvolvido um \textit{Front-end} definitivo para o sistema, junto ao time de \textit{Front-end} da Cadaris, com todos os recursos necessários para uma melhor experiência do usuário.



%\subsection{ORGANIZAÇÃO DESTE RELATÓRIO} %% {ORGANIZAÇÃODO TRABALHO}
%
%As próximas seções estão organizadas como segue:
%
%A seção 2 apresenta as informações detalhadas de todas as atividades realizadas pelo estagiário. A seção está dividida em quatro partes, uma para cada atividade desenvolvida.
%
%A seção 3 descreve as considerações finais em relação a todo o estágio, como: contribuições para a formação, dificuldades encontradas e sugestões para trabalhos futuros.
%
%A seção 4 apresenta os principais problemas observados pelo estagiário em cada etapa e sugestões de melhorias para cada uma delas.
%
%A seção 5 detalha a relação entre as disciplinas cursadas pelo estagiário e as atividades realizadas durante os projetos.
%
%A seção 6 apresenta a conclusão sobre o estágio desenvolvido.



%  \noindent \texttt{Lista de trabalhos realizados:\\
%    ESTÁGIO 1 \\
%    - Reunião sobre as regras de negócio da empresa (DER) \\
%    - Estudo das tecnologias a serem utilizados (Laravel, bootstrap) \\
%    - Dev o sistema para substituir a antes utilizada (antigamente planilhas) \\
%    ESTÁGIO 2 \\
%    - Dev o sistema conforme as regras de negocio atual da empresa (software VBD) \\
%    - Dev o sistema para a geração de relatórios e controle financeiro \\
%    ESTÁGIO 3 \\
%    - Front-end (VueJS) \\
%    - teste, homologacao, e ajuste finais \\
%    - Outras áreas da empresa
%  }

\clearpage



\section{ATIVIDADES DESENVOLVIDAS}

%%--Descrever as atividades realizadas no estágio--%%

Esta seção tem como objetivo detalhar as atividades desenvolvidas durante o estágio supervisionado. Apresentando as atividades de aprendizagem; o desenvolvimento do projeto; e sua aplicação.


\subsection{APRESENTAÇÃO DAS REGRAS DE NEGÓCIO}

A administração de uma empresa é formada por diversos processos administrativos, que podem ser descritos em um conjunto de atividades que podem ser independentes e/ou relacionadas entre si, e que tem o papel de transformar todos os insumos advindos do trabalho em produtos e serviços que atendem as necessidades dos clientes e que são dotados de valor. Em apoio a cada processo são empregados diversas ferramentas tecnológicas. Entretanto, todas as empresas tendem a se evoluírem com o tempo, assim suas regras de negócio e seus processos evoluem juntos, e as suas ferramentas devem ser adaptadas ao novo modelo. 

Em um dado momento na Cadaris, um software que apoia um dos seus principais processos administrativos, especificamente o de Propostas Comerciais, deixou de satisfazer as novas regras de negócio da empresa. Após inúmeras reclamações e \textit{feedbacks} da Cadaris ao proprietário do \textit{software} não serem corretamente atendidas. Então, foi decidido pelo desenvolvimento de um novo sistema exclusivo que possa suportar os novos moldes da empresa, a fim de substituir esta ferramenta defasada.

Portanto, o desenvolvimento deste novo sistema foi a principal responsabilidade dada neste estágio. Iniciando pela sua concepção e formalização, com o apoio de documentações como diagramas e modelos de processos.

O processo de Propostas Comerciais é realizado utilizando um sistema contratado desde 2013, este é instalado localmente em um servidor nas dependências da empresa, para somente acesso local dos funcionários. O sistema foi desenvolvido em \textit{ASP}\footnote{ASP, sigla para \textit{Active Server Pages}, também conhecido atualmente como ASP Clássico, é uma estrutura de bibliotecas básicas (e não uma linguagem) para processamento de linguagens de script no lado servidor para geração de conteúdo dinâmico na Web.} com integração ao banco de dados \textit{Microsoft SQL Server}\footnote{Sistema gerenciador de Banco de dados relacional desenvolvido pela Microsoft.}. E este apresenta muitos problemas, pois não se adapta aos novos padrões de desenvolvimento WEB, assim prejudicando, e muito, a experiência do usuário, por motivos que serão espanados a seguir.

Uma proposta comercial é iniciada com o contato do cliente para o setor de atendimento da Cadaris, no qual, o atendente irá realizar o cadastro de um \textit{job}\footnote{Trabalho, em inglês, trata-se do projeto a ser desenvolvido pela agência.}, preenchendo os dados necessários, em seguida é confeccionado o \textit{briefing}\footnote{O \textit{briefing} é um documento onde constam as informações do cliente com seus requisitos, a descrição do público-alvo e dos objetivos do cliente.} com um cronograma estimado sobre a produção deste \textit{job}. E por fim, o atendimento requisita um orçamento para o departamento responsável pelo financeiro da empresa. 

Um funcionário do financeiro é encarregado de elaborar uma estimativa de custo para a realização deste \textit{job}. A estimativa é composta por dois tipos de custos: Internos e de Operacionalização. A estimativa de Custos Internos é montada com base em horas por tipo de serviço (Planejamento, Atendimento, Redação, Arte, Tráfego e Estratégia). A estimativa de Custos de Operacionalização refere-se à contratação de serviços de terceiros. Também é especificado o prazo e as condições de pagamento.

Logo após, é gerado um documento PDF\footnote{PDF (Portable Document Format) é um formato de arquivo, desenvolvido pela Adobe Systems, para representar documentos de maneira independente do aplicativo.} com esta estimativa e outros dados importantes à respeito do orçamento em que é enviado para a diretoria comercial aprovar e assim retornar para o atendimento. Caso contrário, será devolvido para o financeiro para realizar as alterações exigidas pela diretoria. 

Para concluir a etapa de Proposta Comercial, o atendimento retorna ao cliente com o orçamento, para este ser aprovado por ele, a fim de dar início à produção do \textit{job}. Caso contrário, é solicitado alterações retornando para o \textit{briefing}, ou então o \textit{job} é arquivado se haver desistência do cliente em contratar a agência.

Pode-se ver na Figura a seguir, o modelo de processo da Proposta Comercial.

\textbf{[[ img: modelo de processo Proposta Comercial (pagina em paisagem) ]]}

A etapa de Produção do \textit{job} é realizado pelo setor de produção da Cadaris, utilizando técnicas e outras ferramentas especificas para o serviço desejado, como o \textit{Trello}\footnote{Ferramenta de gerenciamento de projetos em listas extremamente versátil, utilizando o método de \textit{Kanban} para indicar o andamento dos fluxos de produção.} para o gerenciamento do processo de produção. Concluido a produção do \textit{job}, o produto/serviço é entregue ao cliente. E para finalizar, o \textit{job} é encaminhado para o faturamento. 

Também há outras rotinas administrativas que este projeto pretende abranger, já que são realizadas em um processo não automatizado, com o auxílio de planilhas eletrônicas: 


{\singlespacing
\begin{itemize}
  \item Contas a receber;
  \item Contas a pagar;
  \item Controle RH:
  \begin{itemize}
    \item Folha de pagamento;
    \item Salário;
    \item Férias;
    \item Faltas/Atrasos;
    \item Folgas;
    \item Horas Extras;
    \item Vale Refeição;
    \item Vale Transporte;
  \end{itemize}
  \item Controle de Contratos;
  \item Cotação de Fornecedores;
  \item Relatórios:
  \begin{itemize}
    \item Relatório de Comissão;
    \item Relatório de Classificação de Receita;
    \item Relatório de Classificação de Despesa;
    \item Relatório Financeiro (Fluxo de Caixa);
  \end{itemize}
\end{itemize}
}


\subsubsection{PROBLEMAS ATUAIS E O PROJETO DE UM NOVO SISTEMA}

Para compor a Proposta Comercial, este sistema contratado apresenta muitos problemas pelo qual os funcionários devem enfrentar. Resultando em empecilhos para a boa usabilidade e atrapalhando a performance do funcionário. Nesta seção será apontado defeitos do sistema atual e melhorias de recursos necessárias que o novo sistema deverá ter.

O sistema atual tenta ser generalista para suportar inúmeras áreas da empresa e agências com regras de negócio distintas. Assim, muitos de seus módulos se tornam ineficientes para atuar nos novos modelos de negócio, com muitos campos redundantes, ou sem funcionalidade, e não conta com operações automatizas. Um dos problemas mais comuns e sérios que ocorrem é que elementos importantes tem sua relação baseada na inserção manual de códigos identificadores, para realizar a vinculação de um objeto a outro. Assim, de um ponto de vista da experiência do usuário, atrapalha a usabilidade do sistema e facilita o erro humano.

Para o preenchimento do \textit{Briefing}, além de um campo para a escrita livre, são necessários formulários especializados para os tipos de \textit{jobs} que a agência realiza, facilitando e padronizando a confecção destes \textit{Briefings}. O Cronograma é elaborado manualmente, então se faz necessário a incorporação de alguma ferramenta de simples utilização para a elaboração do cronograma estimado, pelo atendente, para a realização do \textit{job}.

Para a transferência do \textit{job} cadastrado de um departamento para outro, durante a passagem das etapas, no sistema atual é necessário realizar um \textit{PIT}\footnote{Sigla para Pedido Interno de Trabalho, é um documento com todas as informações necessárias para solicitar a realização de algum trabalho.}, que é livre para enviar para qualquer pessoa. Mas, visto que a Proposta Comercial da Cadaris segue uma ordem pré-estabelecida de etapas e departamentos envolvidos, então a adoção de um sistema mais especializado é uma ótima solução para agilizar o processo e impedir possíveis erros.

Já para a elaboração da estimativa de custo, em vários quesitos o sistema não atende corretamente ao modelo de negócio da Cadaris. Uma estimativa de custo contempla Custos Internos e Custos de Operacionalização que por sua vez é composto por listas de itens com seus respectivos valores, e para a elaboração desta lista de custos, além de campos desnecessários, há a falta de certas funcionalidades, tais como: maior personalização de taxamentos, arredondamento de valor, subcategorias de Tipo de Serviço no Custo Interno, valores padrão para cada Tipo de Serviço e outras funcionalidades. 

O mais crítico do sistema é que essa estimativa de custo não é diretamente relacionada ao \textit{job}. Então, torna-se necessário preencher campos distintos com a mesma informação, pois o sistema não obtém esta informação do \textit{job} que deveria estar associado a ele. Algumas informações também importantes acabam sendo armazenadas em campos de observações por não ter campos próprios para aquela informação. E para finalizar, a estimativa não carrega as condições de pagamentos dos clientes, já que são distintas para cada um.

Ao final é gerado um documento em PDF contendo esta estimativa, porém faltam informações importantes neles como o prazo e condições de pagamento, e a lista de itens não são corretamente expostas. E no sistema atual este PDF da proposta pode ser gerado mesmo sem a aprovação da diretoria, havendo a necessidade de um bloqueio até a aprovação.

Durante todas estas etapas, é também utilizado no \textit{Trello} um \textit{card} para mostrar o estado do \textit{job}. Mostrando a necessidade de utilizar mais de uma ferramenta para gerenciar o processo de Proposta Comercial.

Para o gerenciamento de outros recursos, como: Funcionários, Departamentos, Clientes e Fornecedores, são utilizados planilhas eletrônicas. Sendo todo os processos relativos a estes recursos feito de modo não automatizado e sem uma centralização dos dados que são relacionados entre si.

Resumindo, a proposta geral deste projeto é o desenvolvimento de um sistema ERP\footnote{Sigla em inglês para \textit{Enterprise Resource Planning}, ou seja, Planejamento dos Recursos da Empresa.}, que deve cuidar de todo o trabalho administrativo e operacional feito na Cadaris, e também automatizar todas estas tarefas antes feitas manualmente. Como por exemplo: faturamento, balanço contábil, fluxo de caixa, administração de pessoal, contas a receber, o dissídio, cálculo de férias e o ponto dos funcionários, e outros.

Há também um módulo especializado em atender a Proposta Comercial e todos os outros processos envolvidos nele.

Outros requisitos importantes são: o sistema deve ser de fácil acesso; possibilidade de utiliza-lo online e fora das dependências da empresa; garantir a disponibilidade; garantir a integridade, a segurança e a centralização de todos os dados; hierarquia de usuários; reduzir ao máximo o custo para desenvolvê-lo e mantê-lo ativo.



%  \noindent TODO: \\
%  - Explicar como eh o Comercial atual (VBD) OK \\
%  - dependeria de outros recursos (func client forn) q usa planilhas e códigos loucos OK \\
%  - automatizar tarefas/rotinas administrativas (ferias, horas, dissidio) OK \\
%  - producao (trello) OK \\
%  - finaceira (contas receber/pagar) OK \\
%  - integracao total dos dados, seguranca, divisao de usuarios e responsabilidades (hierarquia) OK \\




\subsection{ESTUDO DAS TECNOLOGIAS DE DESENVOLVIMENTO WEB}

Nesta subseção é relatada como ocorreu o estudo para a escolha de todas as tecnologias a serem empregadas no desenvolvimento deste sistema WEB, na qual muitas delas o aluno ainda não havia tido contato.

Como pode ser visto nas seções a diante, a UFABC oferece, por meio de disciplinas, todo o embasamento necessário para o desenvolvimento de um sistema, como: teoria da computação; concepção de um sistema de informação; modelamento de banco de dados; conceitos de interface gráfica; paradigmas de programação; programação para web e dispositivos móveis. 

O mercado de trabalho necessita e busca sempre por praticidade, seguindo tendências de desenvolvimento e utilizando o que há de mais novo e eficiente no mercado. Atualmente, há diversos métodos e padrões de projetos que garantem eficiência de uma implementação. Além de inúmeras linguagens de programação, na qual nelas a massiva adoção de \textit{frameworks} para diversos fins.

Um \textit{framework} é um conjunto (ou biblioteca) de classes que se relacionam entre si para disponibilizar ao desenvolvedor funcionalidade especificas. Basicamente um \textit{kit} de ferramentas devidamente implementada, testadas e prontas para o uso. Poupando assim, tempo e trabalho do desenvolvedor de implementar operações básicas como acesso a banco de dados, sistema de templates, mapeamento de rotas, autenticação de usuário e validação de dados.
%  http://www.phpit.com.br/artigos/o-que-e-um-framework.phpit

Com todos os requisitos discutidos anteriormente, escolher corretamente como será formada a arquitetura do sistema foi de total importância. Escolher uma linguagem de programação e com ela um de seus \textit{frameworks} disponíveis, determinará os requisitos mínimos para sua execução, como os custos para desenvolvê-lo e mantê-lo, e viabilizará todos os recursos requisitados. E também é relevante saber seus limites e a possibilidade de utilizar outras bibliotecas para apoiar a necessidade, assim então garantir o sucesso da implementação deste sistema web.

De início foi escolhido a linguagem \textit{PHP}, que será melhor explicada nas seções seguintes, como a linguagem para o \textit{Back-end}, onde será executada em um servidor, afim de servir páginas dinâmicas de \textit{HTML} ao cliente em um navegador de internet. A escolha desta linguagem dar-se pelo fato do discente já ter familiaridade com o desenvolvimento web utilizando \textit{PHP}.

A tarefa seguinte era a pesquisa e escolha de um \textit{framework} para apoiar o desenvolvimento. Após a leitura de blogs e artigos na internet sobre os melhores \textit{frameworks PHP}~[ref:top10]. Após pequenos testes e exames de projetos exemplos que utilizam o \textit{framework}, foi decidido pelo aluno que o \textit{framework} será o \textbf{LARAVEL}. Eleito como o mais popular por diversos sites, como o [ref:sitepoint], o Laravel foi lançado em 2011, mas apesar de ser relativamente novo, tem um enorme ecossistema, com uma ótima documentação e conteúdo didático na internet.
%http://cienciacomputacao.com.br/desenvolvimento/10-melhores-frameworks-php-para-projetos-web/
%https://www.sitepoint.com/best-php-framework-2015-sitepoint-survey-results/

Para o aprender a utilizar o Laravel, o aluno estudou através de vídeo-aulas gratuitas disponíveis na plataforma \textit{LARACASTS}~[ref:laracasts] e sempre consultando a documentação oficial~[ref:Laravel.docs]. Obtendo assim aptidão para criar aplicações web utilizando este \textit{framework}, conhecendo suas funcionalidades e recursos disponíveis.

Para então iniciar a implementação deste sistema ERP.




\subsection{DESENVOLVIMENTO DO SISTEMA} 

  \subsubsection{CADASTROS}
  
  \noindent TODO p/ estagio 1: \\
  Funcionários \\
  - CRUD \\
  - Histórico do funcionário (admissão, promoção, ajuste VR/VT, férias)\\
  - Histórico em Lote\\
  - dissidio automático \\
  - Relatório de faixa Salarial \\
  - Relatório de férias \\
  - Controle de Pontos (Atraso, Faltas(com Atestado), Folgas(Abonadas), Horas Extras) \\
  Departamentos \\
  Usuários \\
  Clientes \\
  - Empresa \\
  - Departamento \\
  - Centros de Custos \\
  Fornecedores \\
  - Empresa \\
  - Centros de Custos 

  
  \subsubsection{PROPOSTA COMERCIAL}
  \noindent TODO p/ estagio 2: \\
  - Listagem \\
  - tipos de Create \\
  - Briefing \\
  - Cronograma \\
  - Orçamento \\
  - cond pgto \\
  - PDFs \\
  - Aprovação da Diretoria \\
  - Aprovação do Cliente \\
  - Produção
  
    
  \subsubsection{RELATÓRIOS E CONTROLE FINANCEIRO}
  \noindent TODO p/ estagio 3: \\
  - Listagem \\
  - tipos de Create \\
  - Briefing \\
  - Cronograma \\
  - Orçamento \\
  - cond pgto \\
  - PDFs \\
  - Aprovação da Diretoria \\
  - Aprovação do Cliente \\
  - Produção


\subsection{INTERFACE E EXPERIÊNCIA DE USUÁRIO}
  \noindent TODO p/ estagio 3: \\
  Front-end


\subsection{HOMOLOGAÇÃO, E AJUSTE FINAIS}

\subsection{OUTRAS ÁREAS DA EMPRESA}

\subsubsection{Resultados}
\subsubsection{Conclusão}

\subsection{DIAGNÓSTICO DOS PRINCIPAIS PROBLEMAS OBSERVADOS E SUGESTÕES DE MELHORIA}



\clearpage
\section{FUNDAMENTAÇÃO TEÓRICA}

%%-- Descrever as disciplinas utilizadas e dar exemplos--%%


As disciplinas cursadas na Universidade Federal do ABC contribuíram de várias formas, direta ou indiretamente, nas atividades realizadas no período de estágio. A Tabela~\ref{tab:ementas} apresenta as disciplinas cursadas pelo aluno e que foram necessárias para o desenvolvimento das atividades. Nas subseções seguintes é descrito também as ferramentas utilizadas.

%   LISTA DE DISCIPLINAS E EMENTA
%  Algoritmo e estrutura de dados
%    Filas
%    Listas
%    Pilhas
%    Ordenação
%  
%  Análise de algoritmos
%    Custo de operações
%  
%  Banco de dados
%    Banco de dados relacional
%    Diagrama Entidade-Relacionamento
%    Diagrama de classes
%    Consultas SQL
%  
%  Bases Computacionais da Ciência
%    Planilhas
%  
%  Engenharia de software
%    Planejamento
%    Modelo Entidade-Relacionamento
%    Padrão MVC
%  
%  Interação Humano-Computador
%    Usabilidade
%    Padrões para interfaces
%    Técnicas de design
%    Ciclo de vida da engenharia de usabilidade
%    Métodos para avaliação da usabilidade
%  
%  Lógica básica
%    Operadores Lógicos
%  
%  Processamento da informação
%    Lógica de programação
%  
%  Programação orientada a objetos
%    Paradigma orientado a objetos
%  
%  Programação para Dispositivos Móveis
%    HTML
%    Responsividade
%  
%  Programação para Web
%    HTML
%    CSS
%    Javascript
%    Modelo MVC
%  
%  Segurança de Dados
%    Criptografia
%    Autenticação
%  
%  Sistemas de Informação
%    Desenvolvimento de sistemas de informação
%    Fundamentos de sistemas
%    Aplicações empresariais
%    ERP - Planejamento dos Recursos Empresariais



\begin{table}[!h] %[!htb]
\centering
\caption{Disciplinas praticadas durante o estágio com os suas respectivas abordagens.}
\label{tab:ementas}

\begin{tabular}{|c|l|}
  \hline
  
  Algoritmo e Estrutura de Dados                       
  &
  \begin{tabular}[c]{@{}l@{}}
    Filas \\
    Listas \\
    Pilhas \\
    Ordenação
  \end{tabular} \\ 
  \hline
  
  Análise de Algoritmos                       
  &
  Custo de operações \\ 
  \hline
  
  Banco de dados                       
  &
  \begin{tabular}[c]{@{}l@{}}
    Banco de Dados Relacional \\
    Diagrama Entidade-Relacionamento \\
    Diagrama de Classes \\
    Consultas SQL
  \end{tabular} \\ 
  \hline
  
  Bases Computacionais da Ciência                       
  &
  Planilhas eletrônicas \\ 
  \hline
  
  Engenharia de Software                       
  &
  \begin{tabular}[c]{@{}l@{}}
    Planejamento \\
    Modelo Entidade-Relacionamento \\
    Padrão MVC
  \end{tabular} \\ 
  \hline 
  
  Interação Humano-Computador                       
  &
  \begin{tabular}[c]{@{}l@{}}
    Usabilidade \\
    Padrões para Interfaces \\
    Técnicas de Design \\
    Ciclo de Vida da Eng. de Usabilidade \\
    Métodos para Avaliação da Usabilidade
  \end{tabular} \\ 
  \hline 
  
  Lógica Básica                   
  &
  Operadores Lógicos \\ 
  \hline 
  
  Processamento da informação                       
  &
  Lógica de programação \\ 
  \hline 
  
  Programação Orientada a Objetos                       
  &
  Paradigma Orientado a Objetos \\ 
  \hline 
  
  Programação para Dispositivos Móveis  
  &
  \begin{tabular}[c]{@{}l@{}}
    HTML \\
    Responsividade
  \end{tabular} \\ 
  \hline 
  
  Programação para Web                       
  &
  \begin{tabular}[c]{@{}l@{}}
    HTML \\
    CSS \\
    Javascript \\
    Modelo MVC
  \end{tabular} \\ 
  \hline
  
  Segurança de Dados                       
  &
  \begin{tabular}[c]{@{}l@{}}
    Criptografia \\
    Autenticação
  \end{tabular} \\ 
  \hline
  
  Sistemas de Informação                       
  &
  \begin{tabular}[c]{@{}l@{}}
    Fundamentos de SI \\
    Desenvolvimento de SI \\
    Aplicações Empresariais \\
    ERP
  \end{tabular} \\ 
  \hline
  
\end{tabular}
\end{table}



\subsection{ENGENHARIA DE SOFTWARE}
Garantir que um \textit{software} seja bem sucedido é necessário seguir especificação descritas pela Engenharia de Software, para assim melhor atender as necessidades dos usuários e prevenir futuras falhas.

Como descrito por Pressman~[ref:Pressman], ter o planejamento completo do projeto antes da própria implementação é de extrema importância. Com um bom planejamento, o sistema reduzirá drasticamente o impacto de suas falhas, enquanto o \textit{software} estiver em larga utilização.

Inúmeras especificação foram abordadas em aulas e agora o discente as colocou em prática ao desenvolver este sistema. Desde as reuniões iniciais na empresa, as técnicas aprendidas em aula foram aplicadas, com o apoio de diagramas e modelos para documentar o \textit{software} a ser implementado.


\subsection{DESENVOLVIMENTO WEB}
% https://pt.wikipedia.org/wiki/W3C
% https://pt.wikipedia.org/wiki/Aplica%C3%A7%C3%A3o_web
% https://pt.wikipedia.org/wiki/Front-end_e_back-end
Como um dos principais requisitos é do sistema ser \textit{online}, criar um \textit{WebApp} pode ser considerado complexo, pois utilizam diferentes tecnologias para criar uma aplicação semelhante ao de \textit{desktop}, porém em um navegador de internet. Esta aplicação é executada em um servidor, que receberá requisições do cliente e retornará respostas a ele conforme as ações desejadas. Já no navegador, haverá a exibição das informações necessárias e suportará as interações do usuário. E todos estes fundamentos foram apresentados durante a disciplina de Programação para Web.

Assim podemos separar uma \textit{WebApp} em duas camadas: \textit{Back-end} e \textit{Front-end}. Na qual o \textit{Front-end} é o responsável por interagir com o usuário, afim de coletar os nados necessários e transmiti-los ao \textit{Back-end}, para que estes dados possam ser processados, afim de retornar uma resposta.

Nas subseções a seguir, será exposto as tecnologias utilizadas.
 

\subsubsection{BACK-END}
Do lado do servidor, sua principal função eh receber requisições e devolver respostas. Bem como tambem ter acessoa o banco de dados e realizar todos o processamento de dados necessários.

\subsubsubsection{PHP}
% https://pt.wikipedia.org/wiki/PHP
O PHP\footnote{PHP é um acrônimo recursivo para \textit{PHP: Hypertext Preprocessor}, originalmente \textit{Personal Home Page}.} é uma linguagem interpretada capaz de gerar uma página web a ser visualizada no lado do cliente. Suas principais características são sua velocidade e robustez, orientada a objetos, independente de plataforma e tipagem dinâmica. Nela há varias extensões que possibilita o PHP se conectar com outras ferramentas, como: Bancos de Dados, \textit{socket}, geração de PDF, OpenSLL e outros.




\subsubsubsection{LARAVEL}
% https://en.wikipedia.org/wiki/Laravel
Laravel é um \textit{framework} PHP livre e \textit{open-source} para o desenvolvimento de sistemas web, utilizando o padrão MVC\footnote{Sigla para \textit{Model-view-controller}, em português modelo-visão-controlador, é um padrão de arquitetura de software que separa em camadas com suas respectivas responsabilidades.}. O Laravel possui recursos e características que sobressaem dos demais \textit{framework}, como: 

{\singlespacing
\begin{itemize}
  \item Sua sintaxe simples e concisa;
  \item Um sistema modular com gerenciador de dependências dedicado;
  \item Várias formas de acesso a banco de dados relacionais
  \item Vários utilitários indispensáveis no auxílio ao desenvolvimento e manutenção de sistemas:
  \item \textit{Eloquent ORM} para mapeamento objeto-relacional;
  \item \textit{Query Builder} é um conjunto de classes e métodos capaz de criar consultas em forma de programação;
  \item Roteamento de links;
  \item Controladores \textit{Restful};
  \item Carregamento automático de classes sob demanda;
  \item Compositores de \textit{View}, como o \textit{Blade} para compor \textit{view} e fornecer estruturas de controle, utilizando o cache para melhor desempenho;
  \item \textit{Database Migration} para versionar a estrutura do banco de dados, facilitando o desenvolvimento e a manutenção do sistema;
  \item \textit{Database seeding} para popular o banco de dado com dados a serem utilizados em testes.
  \item Paginação automática de listagens de dados;
  \item \textit{Form request} para validação da entrada de dados vindo do cliente;
  \item \textit{Artisan CLI} é uma interface de linha de comandos para automatizar tarefas, como a migração do banco de dados ou criação de componentes do Laravel, ajudando a melhor gerir o \textit{framework};
  \item Código fonte do Laravel está hospedado no GitHub;
  \item Licença MIT.
\end{itemize}
}



\subsubsection{FRONT-END}

\subsubsubsection{HTML}
% https://pt.wikipedia.org/wiki/HTML

\subsubsubsection{JAVASCRIPT}
% https://pt.wikipedia.org/wiki/JavaScript

\subsubsubsection{CSS}
% https://pt.wikipedia.org/wiki/Cascading_Style_Sheets


\subsection{BANCO DE DADOS}
\subsubsection{DER}
\subsubsection{MYSQL}
% https://pt.wikipedia.org/wiki/MySQL

\subsection{SISTEMAS DE INFORMAÇÃO}

\subsection{INTERAÇÃO HUMANO-COMPUTADOR}

\subsection{FERRAMENTAS}
\subsubsection{GIT}
\subsubsection{PHPSTORM}




\subsection{PLANEJAMENTO DE PROJETOS}


Segundo os pontos citados na Engenharia de Software e nos livros de Somerville \textit{[15]} e Pressman \textit{[16]}, não basta apenas programar e desenvolver projetos, é necessário ter um planejamento completo antes de iniciar a programação em si, além de possuir desenvolvedores competentes na equipe. Todo sistema está sujeito a falhas, mas quando há bom planejamento, incluindo os imprevistos, dificilmente terá grandes problemas durante o período de utilização, nada que o comprometa seriamente. Tais fatos foram notados ao desenvolver os sistemas de Bolsa de Estudos PROMAIS e Sistema de Gestão de Atividades. O primeiro foi totalmente planejado, desde a estrutura das classes como quais ferramentas seriam utilizadas e o projeto foi desenvolvido do começo ao fim como o previsto. O segundo projeto havia sido planejado, mas durante o desenvolvimento foi notada muita repetição de códigos e a necessidade de uma ferramenta que tornasse a aplicação mais dinâmica e suprisse esse obstáculo, e assim foi pesquisado e encontrado o AngularJS \textit{[4]}.




\subsection{DESENVOLVIMENTO WEB}

\subsubsection{HTML}

Em \textbf{Programação para Web} foram desenvolvidas aplicações para web e em \textbf{Programação para Dispositivos Móveis} foram vistos conceitos semelhantes ao HTML. O HTML5 é uma linguagem que foi desenvolvida para exibir conteúdos em páginas web. A diferença para algumas outras linguagens de programação é que o HTML5 não necessita ser compilado, ele é apenas interpretado pelo navegador. A programação é realizada através de tags (<tag>) que podem possuir atributos e elementos filhos, desde que estejam dentro da tag de fechamento, que é representada por uma barra (</tag>). A Figura 3.1 demonstra a estrutura de um arquivo HTML com algumas tags novas que foram introduzidas na versão 5, como header, nav, section, article, aside e footer. Novos elementos foram introduzidos para dar semântica à linguagem, ou seja, ao ler um código é possível ter noção do conteúdo que será apresentado de acordo com as tags utilizadas, nas versões anteriores, todo o conteúdo possuía o mesmo nível de informação.

\begin{figure}[!h]
\centering
\includegraphics[width=0.6\textwidth]{figura31}
\caption{Estrutura do HTML com novas tags introduzidas na versão 5.}
\end{figure}

Com base na Figura 3.1, é demonstrada a herança de tags. As tags que são apresentadas dentro de uma tag de abertura e fechamento, são consideradas filhas, como h1, form e nav que são filhas da tag header.

O HTML5 é responsável pela exibição de conteúdo, sendo assim, não é capaz de criar novos elementos, conectar com um banco de dados ou com algum servidor, por exemplo. Essas outras atividades são realizadas por outras linguagens de programação que caminham em conjunto com o HTML5.



\subsubsection{JAVASCRIPT}


A linguagem de programação Javascript permite que haja interação entre o conteúdo HTML5 e o usuário, tornando a conteúdo da aplicação dinâmico. O DOM é uma multiplataforma que representa como as marcações em HTML, XHTML e XML são organizadas e lidas pelo navegador. Uma vez indexadas, estas marcações se transformam em elementos de uma árvore que você pode manipular via API, ou seja, o Javascript é capaz de manipular essas informações. Após o HTML ser carregado e interpretado pelo navegador, ele mesmo não tem poder para alterá-lo, mas o Javascript pode inserir, alterar ou remover nós em tempo de execução (Figura 3.2) \textit{[17]}.

\begin{figure}[!h]
\centering
\includegraphics[width=1\textwidth]{figura32}
\caption{Estrutura de uma árvore DOM.}
\end{figure}

A Figura 3.3 apresenta um exemplo de manipulação de uma árvore DOM com o Javascript. A primeira linha cria uma tag do tipo “div” e salva na variável “el”. A segunda linha acrescenta a variável “el” ao elemento “body”. A terceira linha acrescenta uma classe CSS
“container” ao elemento “el”. A quarta linha adiciona uma margem superior de 30px ao elemento.
Se esse código fosse desenvolvido diretamente pelo HTML5, seria o mesmo que:
<body><div class=”container” style=”margin-top: 30px”></div></body>

\begin{figure}[!h]
\centering
\includegraphics[width=1\textwidth]{figura33}
\caption{Exemplo de manipulação de DOM com Javascript.}
\end{figure}


\subsubsection{ANGULARJS}

O AngularJS \textit{[4]} foi o grande encarregado por essa função de manipular o DOM nos sistemas de Bolsa de Estudos PROMAIS e o de Gestão de Atividades. A única necessidade do desenvolvedor é se preocupar com o JSON gerado e o AngularJS \textit{[4]} o interpreta e cria o DOM de acordo com as informações que recebe.
A Figura 3.4 é uma implementação básica em AngularJS \textit{[4]} para a criação de um novo nó com o conteúdo desejado. O primeiro bloco é o código HTML em que a tag “li” possui uma instrução “ng-repeat” do AngularJS \textit{[4]} para adicionar o conteúdo “text” da lista “opts” enquanto houver filhos. O código abaixo é o Javascript responsável por alimentar o JSON e adicionar o valor “x” ao clicar no botão “Add”. Á direita está o resultado final da aplicação de exemplo.
Esse mesmo código em Javascript puro seria um pouco mais complexo, mas nesse caso foi necessário acrescentar um novo valor no JSON e a aplicação se encarrega de acrescentar um novo nó automaticamente.

\begin{figure}[!h]
\centering
\includegraphics[width=1\textwidth]{figura34}
\caption{Exemplo de criação de nó com AngularJS.}
\end{figure}

\subsubsection{CSS}

O HTML5 possui atributos para personalizar as tags adicionadas, porém é um método depreciado e pouco eficiente com o surgimento do CSS3.

O CSS3 é uma linguagem de folha de estilo utilizada para definir a apresentação de documentos escritos em uma linguagem de marcação, como HTML5 e tem como papel principal separar o conteúdo de um documento e o seu formato.

A Figura 3.5 apresenta duas maneiras de como inserir personalizações CSS a um elemento: por classe e inline. A classe pode ser atribuída a vários elementos e assim todos terão apresentação em comum. Ao alterar uma classe, todos os elementos serão afetados e terão sua apresentação modificada. Os atributos da classe podem ser escritos no mesmo arquivo ou em um arquivo externo e importado. O método inline é adicionado com o “style”, é atribuído apenas ao elemento em questão e, caso haja alguma classe no elemento, sobrepõe os atributos coincidentes da classe.

\begin{figure}[!h]
\centering
\includegraphics[width=1\textwidth]{figura35}
\caption{Métodos de utilização do CSS \url{http://bit.ly/1G00zSp}.}
\end{figure}


\subsection{LÓGICA DE PROGRAMAÇÃO}
\subsubsection{CONEXÃO COM BANCO DE DADOS}

Todo o conteúdo até o momento está do lado do cliente em JSON, não foi enviado nada para o servidor e não foi feita nenhuma persistência de informações, portanto as informações são perdidas ao fechar a aplicação. Para a conexão com o banco de dados foi utilizada a linguagem C\# da Microsoft, ela é uma linguagem orientada à objetos, fortemente tipada, foi baseada no C++ e possui similaridades com Java, linguagem desenvolvida em \textbf{Processamento da Informação, Programação Orientada a Objetos} e conceitos vistos em \textbf{Lógica Básica, Algoritmos e Estrutura de Dados e Análise de Algoritmos}.

A Figura 3.6 é um exemplo de conexão básica com o banco de dados na linguagem C\#
em que a conexão é declarada, em seguida são adicionados os parâmetros da conexão, a conexão é aberta, uma mensagem é exibida, a conexão é fechada e outra mensagem é exibida.

\begin{figure}[!h]
\centering
\includegraphics[width=1\textwidth]{figura36}
\caption{Exemplo de conexão básica com o bando de dados em C\#.}
\end{figure}


\subsection{BANCO DE DADOS}

Os bancos de dados de todas as aplicações foram desenvolvidos em SQL Server da Microsoft e a disciplina \textbf{Banco de Dados} instruiu os primeiros passos para desenvolver uma estrutura sólida e coerente, de acordo com as formas normais. A disciplina de\textbf{Segurança de Dados} também contribuiu para que as informações fossem transferidas de forma criptografada e segura de usuários não autorizados.

Com banco de dados é possível criar consultas específicas para retornar qualquer tipo de informação necessária pelo usuário. Tabelas podem ser relacionadas através das chaves primárias e estrangeiras. As chaves primárias representam uma identificação única para a informação e a chave estrangeira é a representação de uma informação de outra tabela. A Figura 3.8 apresenta o fluxo de uma consulta e os elementos possíveis, como: colunas a serem exibidas, as tabelas a serem consultadas, as condições de filtros, agrupamentos e filtros após os resultados.

\begin{figure}[!h]
\centering
\includegraphics[width=0.7\textwidth]{figura38}
\caption{Modelo de consulta em banco de dados.}
\end{figure}

A modelagem dos bancos desenvolvidos pode ser vista em Figura 2.6 e Figura 2.14.
Nelas é possível notar que estão nas formas normais, existem relacionamentos “um para um”,
“um para n” e “n para n”.

Antes do desenvolvimento prático, foram feitas reuniões e modelagens conceituais para que fosse possível chegar ao modelo ideal sem que houvesse modificações durante o desenvolvimento das aplicações.



\clearpage
\section{CONSIDERAÇÕES FINAIS}
%%-- Conclusão do relatório, enfatizar os resultados a contribuição para a formação, dificuldades e futuras melhoras --%%


Este capítulo tem por objetivo descrever as contribuições que o estágio proporcionou à formação acadêmica e profissional, as dificuldades encontradas durante o processo e sugestões de trabalhos futuros.


\subsection{CONTRIBUIÇÕES PARA A FORMAÇÃO}

O estágio exigiu conhecimento e aplicação de várias disciplinas como Processamento da Informação, Programação para Web, Programação Orientada a Objetos, Banco de Dados, Lógica Básica, Computadores, Ética e Sociedade, Linguagens Formais e Autômata, Segurança de Dados, Programação para Dispositivos Móveis, Análise de Algoritmos e Engenharia de Software. O estágio exigiu o conhecimento para criar lógicas que solucionassem problemas,
operadores lógicos, expressões regulares, consultas em banco de dados, desenvolvimento em camadas (MVC) e planejamento.

As ferramentas ensinadas ao aluno durante os cursos de formação facilitaram as atividades e exigiram uma curva de aprendizado menor nos assuntos em que foi necessário aprofundamento, considerando que o aprendizado em um ambiente acadêmico acontece em um cenário ideal e pequeno em relação ao mercado de trabalho.

O período total foi suficiente para perceber a diferença entre a teoria e a prática. Na teoria todo o assunto é passado ao aluno, os conceitos, maneiras de resoluções de problemas, os exercícios e projetos são grandes desafios e são os que põem a prova tudo o que foi aprendido nas aulas, porém, quando o aluno se depara com a realidade do mercado de trabalho, nota a diferença da dimensão de tudo o que foi visto antes e que nas aulas, apesar de parecer um projeto grande, é apenas um pequeno exemplo que pode acontecer na realidade. Essa distância entre a teoria e a prática é o que amadurece e faz com que o estagiário aplique tudo o que foi aprendido, contribuindo no crescimento profissional.


\subsection{DIFICULDADES ENCONTRADAS}

O desenvolvimento de aplicações no ambiente de produção exige muita disciplina do profissional. O código deve ser bem estruturado, pois será mantido por uma equipe. Deve haver um planejamento do projeto antes de iniciar qualquer programação para evitar erros e trabalhos desnecessários. O profissional necessita pesquisar muito sobre como realizar a tarefa, qual seria a melhor solução, quais seriam as melhores ferramentas, conversar com outros desenvolvedores para adquirir experiência e criar uma solução ideal para o problema.

Algumas ferramentas utilizadas no decorrer do estágio não foram ensinadas nos cursos de formação do aluno, dificultando inicialmente o desenvolvimento dos projetos. Porém, tal fato implicou em um maior empenho e dedicação em estudos e pesquisas, gerando ótimos resultados nos projetos e amadurecendo o aluno profissionalmente.


\subsection{SUGESTÕES DE TRABALHOS FUTUROS}

As disciplinas ministradas aos alunos devem possuir uma reciclagem periodicamente,
pois a área de tecnologia possui novidades constantemente. Além da reciclagem nos assuntos de cada disciplina, também é preciso verificar a necessidade de novas disciplinas que possam abranger as novidades de hardware e/ou software.

Os projetos desenvolvidos pelo aluno possuem tecnologias e ferramentas recentes e esse caminho deve ser mantido. Foram utilizadas ferramentas de código aberto, que facilita a atualização por parte dos desenvolvedores e da comunidade que participa com constantes contribuições. Outras ferramentas e recursos podem ser agregados aos sistemas para facilitar a experiência do usuário, mas com cuidado para que não haja exageros e conflitos com as ferramentas atuais.

O AngularJS demonstrou ser uma ferramenta com pequena curva aprendizado para tarefas mais comuns em um sistema. Ele deve ser mantido e aperfeiçoado nos sistemas atuais ecotado para o desenvolvimento de novos sistemas.


\clearpage
\section{REFERÊNCIAS BIBLIOGRAFICAS}
%%-- Fontes utilizadas no relatório --%%

%%-- Indica-se o uso da bilioteca bibTEX--%%


[1] Tripletech, \url{http://www.tripletech.com.br/}, acessado em 03/08/2015 às 06h50

[2] Faculdade de Direito de São Bernardo do Campo, http://www.direitosbc.br/, acessado em 03/08/2015 às 07h00.

[3] Faculdade de Direito de São Bernardo do Campo - Bolsas de Estudos,

\url{http://www.direitosbc.br/a-faculdade-servicos-bolsa-de-estudo.aspx}, acessado em 04/08/2015 às 07h25.

[4] AngularJS by Google, \url{https://angularjs.org/}, acessado em 04/08/2015 às 07h40.

[5] JSON, \url{http://json.org/}, acessado em 04/08/2015 às 07h45.

[6] Bootstrap, \url{http://getbootstrap.com/}, acessado em 04/08/2015 às 08h15.

[7] Runrun.it, \url{http://runrun.it/}, acessado em 04/08/2015 às 13h20.

[8] Angular Material Design by Google,\url{ https://material.angularjs.org/latest/\#/}, acessado em 05/08/2015 às 13h50.

[9] Google, \url{http://google.com.br/}, acessado em 05/08/2015 às 13h55.

[10] Android 5.0 Lollipop,

\url{https://www.android.com/intl/pt-BR_br/versions/lollipop-5-0/},
acessado em 06/08/2015 às 07h35.

[11] Especificações Material Design by Google, https://www.google.com/design/spec/ material design/introduction.html, acessado em 06/08/2015 às 07h40.

[12] Google Inbox, \url{http://inbox.google.com.br/}, acessado em 06/08/2015 às 07h45.

[13] Google Music, \url{http://music.google.com.br/}, acessado em 06/08/2015 às 07h46.

[14] Google Keep, \url{http://keep.google.com.br/}, acessado em 06/08/2015 às 07h47.

[15] Sommerville, I. Engenharia de Software. 8.ed. - São Paulo : Addison-Wesley, 2007.

[16] Pressman, Roger S. Engenharia de Software. 6.ed. - Rio de Janeiro: McGraw-Hill, 2006

[17] O que é DOM, \url{http://tableless.com.br/tenha-o-dom/}, Acessado em 06/08/2015 às 13h35

\end{document}
